\documentclass[11pt]{article}
\usepackage{verbatim}
\usepackage{amsmath}
\usepackage{amssymb}
\usepackage{graphicx}
\graphicspath{{figures/}}
% use \Cpp, not C++
\def\Cpp{{C\nolinebreak[4]\hspace{-.05em}\raisebox{.4ex}{\tiny\bf ++}}\:}

\usepackage{hyperref}

\setlength{\parskip}{3mm}
\setlength{\headheight}{14pt}
\newcommand{\Matrix}[1]{\boldsymbol{#1}}
\newcommand{\Vector}[1]{\boldsymbol{#1}}
% \Eqn{} allows for different styles of equation references:
\newcommand{\Eqn}[1]{Eq.\ \ref{#1}}  % AIAA
\begin{document}
\title{\Huge Design and Evolution of Flaps}
\author{Edward E. Meyer}
\maketitle

Flaps is the evolution of a program I wrote at Boeing called Apex,
which in turn is the evolution of a program I wrote for my thesis on
flutter of a Darrieus wind turbine.
The common thread here is the use and development of a numerical continuation
method: pseudo-arclength minimum-norm corrector.
% There have been many formulations of the flutter equations over the years;
% some have been called ``flutter methods'' but the formulation used can
% often be solved with continuation, for example the
% p-k method \cite{hassig1971approximate},
% the p-method \cite{hassig1971approximate}, and
% the g-method \cite{chen2000damping}.
After more than 30 years solving flutter equations with numerical continuation
I feel confident that it is the most appropriate and versatile method available.

\section{Thesis}
Computing flutter on the Darrieus wind turbine using the traditional
V-g formulation (aka k-method) showed some strange behavior, probably
due to the gyroscopics; that and the need to connect the dots led me
to implement a continuation method so that I could try the p-k
method \cite{hassig1971approximate} (aka British method).
The p-k formulation proved to be
much more realistic, without mode switching which I saw with V-g.
In addition to these advantages I realized I could use the same code
to do parameter studies like stiffness or mass variations with only
minor changes to the code.
% I hired into the Flutter Research group in Boeing in 1978, thinking
% I was almost finished with my thesis; it wasn't until 1982 that I
% finished, partly due to developing the flutter solution technique
% which became an appendix.
% Especially helpful was Charles Pratt-Barlow, who encouraged me to implement
% an improved flutter solution technique.

\section{Apex}
I hired into the Flutter Research group in Boeing in 1978. At the time it was
led by M.J. Turner, known by some as the father of the finite-element method.
Flutter Research was a group of 5-6 engineers and 3-4 programmers.
I was hired to maintain the flutter code and develop new capabilities.

The flutter code was part of a larger program called ATLAS, which did
analyses including finite-element modeling, stress, vibration, dynamic loads,
and flutter. There was not much to maintain in the flutter code since it
had few capabilities, mainly the solution of a complex eigenvalue problem,
the result of a V-g formulation of the flutter equation.
A V-g formulation assumes that adding structural damping is a close
approximation to the true damping so that the problem becomes an
eigenvalue problem, which has robust solution techniques;
unfortunately there is no reliable way to connect solutions so the
engineer was reduced to ``connecting the dots''.

% The program would compute frequency and structural damping at a given
% set of reduced frequencies and the engineer would connect the dots by hand.
I was asked to develop the p-k flutter solution in ATLAS, which
gives the same flutter points but more accurate damping away from flutter.
The only published p-k solution method at the time was to formulate
the problem as a nonlinear eigenvalue problem and compute eigenvalues
by determinant iteration:
solving the nonlinear eigenvalue problem by searching for places where
the determinant of the dynamic matrix is zero \cite{hassig1971approximate}.
But because determinants are extremely large or small (more so as the
problem size increases), this is a poor choice for modern analyses.

I chose to take a crude continuation code from my thesis
\cite{meyer1982aeroelastic} and adapt
it to ATLAS. In doing that I also added some simple parameter variation capability
which proved to be much more popular than the p-k capability since
it saved even more time than just connecting the dots with p-k.

Around this time some open-source continuation codes were appearing;
I chose to replace the code from my thesis with one called PITCON
which has a very good stepsize algorithm \cite{rheinboldt1983algorithm}.
PITCON is written in Fortran 77 so eventually I translated it to C, then
\Cpp. In re-writing it I changed the corrector from locally-parameterized
with LU factorization to minimum-norm with QR factorization, which requires
slightly fewer iterations and uses the more stable QR factorization.

An early request was to include controls equations using equations
coded by the users in Fortran. The difficulty was computing derivatives
of the equations with respect to, for example frequency and velocity.
The alternatives were to require the user to supply derivatives, use
finite-differencing, or a programming technique called
``automatic differentiation''.
To save the user the effort of coding the derivatives, and to
avoid the inaccuracies of finite-differencing, I implemented a system
that transformed the user's equation to \Cpp where I could use automatic
differentiation to get exact derivatives along with the equations.

One of the advantages to treating flutter equations as a parameterized
system of nonlinear equations is that this is the most general way of
looking at these equations.
As a result it has been relatively easy to add new capabilities, such as
controls equations, structural nonlinearities, and new formulations of the flutter
equation
\cite{hassig1971approximate},
\cite{chen2000damping}, \cite{edwards2008flutter}.
I believe that has been the key to its success.

\section{Flaps}
When I retired from Boeing in 2012 I began writing papers about
aspects of flutter I'd been pondering but never had time to pursue
at Boeing. In order to try some of these ideas, try new
programming techniques, and create examples for papers, I re-wrote Apex
and called it Flaps.

Among the features a complete re-write allowed were multiple nonlinearities,
new types of matrix parameterization, and multiple-threaded
(parallel) mode tracking.
To someone who used Apex before I retired, Flaps probably
looks familiar; this was intentional in case Boeing decided to try it.
It was, however, written from scratch.

In order to be able to create test cases I wrote a simple beam
finite-element code with double-lattice unsteady aerodynamics. I used
the Goland wing to verify the code and build a few other test cases.
It uses branch-mode substructuring to create models with discrete
degrees-of-freedom for nonlinear test cases, and component-mode
synthesis to compare with branch-modes.

One feature of Flaps that is different from Apex is controls equations.
Controls equations are now written by the user in \Cpp instead of
Fortran. This is because of an effort to make more use of user-written
custom code to modify any matrix. It would not be hard to allow
Fortran controls equations, I just haven't done it.
% Controls equations are now treated like any other custom code
% One difference between Flaps and Apex that would effect a


By the time I retired Apex was written almost entirely in \Cpp and
the move to \Cpp continued with Flaps, using some of the new features
of \Cpp -11.
Many of the new features in \Cpp -11 allow for cleaner, easier to
understand code. Others are more for better performance, for instance
concurrency.

Concurrency (also known as parallel programming) is the ability to run
parts of a program simultaneously using threads. Modern PCs can run
parts of a program in each of its threads; my desktop PC can run 16
threads. There are other ways to use threads but \Cpp -11 makes it easy
without needing extra libraries.
Because tracing a mode with continuation can be done independently of
all the other modes, converting Flaps to use threads was straightforward.

\section{Papers}
The first paper I wrote in retirement \cite{meyer2014unified} showed how
continuation methods could be used for a wide variety of flutter
calculations. The second one \cite{meyer2015continuation}
had its roots years earlier when a new control-law became singular
at a certain flight condition
in Apex, causing the program to quit tracking a mode.
At the time I didn't know how to treat the problem and fortunately it
never showed up in flight. I now realize it was a bifurcation so this
paper showed how it could be a standard check in the code.
Next I wrote a paper showing how a mathematical technique called
Topological Degree could be used to determine if flutter is present
in a given range of parameters \cite{meyer2015topological}.
Then a paper showing how multiple nonlinearities could be treated with
describing functions and continuation \cite{meyer2016continuation},
followed by one showing how these techniques could be used to find
something I called Latent LCO \cite{meyer2021latent}.

\section{What's next?}
I've been trying different types of nonlinearities and automating
the process of deriving describing functions, some of which I've added
to Flaps. I'm  also looking at the possibility of using Flaps to
compute nonlinear flutter of very flexible aircraft \cite{patil2001limit}.

Something I've been interested in for a long time is how to
take into account uncertainties in the terms of the flutter equation.
There has been some work on this based on control theory with uncertainties
\cite{danowsky2010evaluation}, \cite{iannelli2017nonlinear}, \cite{dai2014methods},
\cite{bueno2015flutter},
\cite{borglund2004mu}.
These treat the flutter equation as a control system, losing physical
feel for the problem. I have experimented with using interval methods
\cite{neumaier1990interval} without much success so far; I intend to
continue.

But my overriding goal with Flaps is to promote the use of continuation
methods for solving flutter equations. To this end I plan on giving the
code to anyone who could further this goal.

% \begin{thebibliography}{}
\bibliography{../flaps}
\bibliographystyle{acm}
% \bibliographystyle{aiaa}
% \bibliographystyle{../bib/bst/mdpi}


% \end{thebibliography}
\end{document}

