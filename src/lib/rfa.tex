\documentclass[11pt]{article}
\usepackage{ifthen}
\usepackage{calc}
\usepackage{verbatim}
\usepackage{amsmath}
\setlength{\parskip}{3mm}
\setlength{\headheight}{14pt}
\newcommand{\Matrix}[1]{\boldsymbol{#1}}
\begin{document}
\title{\large Rational Function Approximation}
\author{Edward E. Meyer}
\maketitle

This approximation,
due to Roger \cite{roger}
uses a least-squares fit of a certain rational-polynomial
to a set of complex matrices.
Given a set of
$m$
unsteady aero matrices at various values of complex
reduced frequency
$p$

\begin{equation}
\Matrix{Q}_k = \Matrix{Q}(p_k) \; \; \; \; k = 1, m
\end{equation}

where
$p = \frac{s}{v_t}$
is the complex reduced-frequency,
$s$ is the Laplace variable, and $v_t$ is the true airspeed,
approximate $\Matrix{Q}$ with an analytic function of $p$.
Roger's approximation to the aero matrix has the form

\begin{equation}
  \Matrix{Q}(p) \approx \Matrix{R}_0 + p\Matrix{R}_1 + p^2 \Matrix{R}_2 +  \sum_{i=1}^{\ell} \frac{p}{p + \beta_i} \Matrix{R}_{i+2}
\end{equation}

where the matrices $\Matrix{R}_i$ are real, and the $\beta_i$ are real.

\par
To determine the matrices $\Matrix{R}_i$, write $m$ equations, one
for each aero matrix $\Matrix{Q}_k$:

\begin{displaymath}
\left[
\begin{array}{cccccc}
1 & p_1 & p_1^2 & \frac{p_1}{p_1 + \beta_1} & \ldots & \frac{p_1}{p_1 + \beta_{\ell}} \\
1 & p_2 & p_2^2 & \frac{p_2}{p_2 + \beta_1} & \ldots & \frac{p_2}{p_2 + \beta_{\ell}} \\
  &     &       &                           & \ddots &                           \\
1 & p_m & p_m^2 & \frac{p_m}{p_m + \beta_1} & \ldots & \frac{p_m}{p_m + \beta_{\ell}}
\end{array}
\right]
\left[
\begin{array}{cccc}
r_{0,11}   & r_{0,21}   & \ldots & r_{0,nn} \\
r_{1,11}   & r_{1,21}   & \ldots & r_{1,nn} \\
           &            & \ddots &          \\
r_{\ell+3,11} & r_{\ell+3,21} & \ldots & r_{\ell+3,nn}
\end{array}
\right]
\end{displaymath}

\begin{equation}
 =
\left[
\begin{array}{cccc}
q_{0,11}   & q_{0,21}   & \ldots & q_{0,nn} \\
q_{1,11}   & q_{1,21}   & \ldots & q_{1,nn} \\
           &            & \ddots &          \\
q_{m,11}   & q_{m,21}   & \ldots & q_{m,nn}
\end{array}
\right]
\end{equation}

where
$r_{k,ij}$
is the $(i,j)$
element of the real
$(n,n)$
matrix $\Matrix{R}_k$
and $q_{k,ij}$
is the
$(i,j)$
element of the complex
$(n,n)$
matrix $\Matrix{Q}_k$.

\par
Expanding the complex terms into real and imaginary,

\begin{displaymath}
\left[
\begin{array}{cccccc}
1 & \Re(p_1) & \Re(p_1^2) & \Re(\frac{p_1}{p_1 + \beta_1}) & \ldots & \Re(\frac{p_1}{p_1 + \beta_{\ell}}) \\
0 & \Im(p_1) & \Im(p_1^2) & \Im(\frac{p_1}{p_1 + \beta_1}) & \ldots & \Im(\frac{p_1}{p_1 + \beta_{\ell}}) \\
1 & \Re(p_2) & \Re(p_2^2) & \Re(\frac{p_2}{p_2 + \beta_1}) & \ldots & \Re(\frac{p_2}{p_2 + \beta_{\ell}}) \\
0 & \Im(p_2) & \Im(p_2^2) & \Im(\frac{p_2}{p_2 + \beta_1}) & \ldots & \Im(\frac{p_2}{p_2 + \beta_{\ell}}) \\
  &          &            &                                & \ddots &                           \\
1 & \Re(p_m) & \Re(p_m^2) & \Re(\frac{p_m}{p_m + \beta_1}) & \ldots & \Re(\frac{p_m}{p_m + \beta_{\ell}}) \\
0 & \Im(p_m) & \Im(p_m^2) & \Im(\frac{p_m}{p_m + \beta_1}) & \ldots & \Im(\frac{p_m}{p_m + \beta_{\ell}}) \\
\end{array}
\right]
\left[
\begin{array}{cccc}
r_{0,11}   & r_{0,21}   & \ldots & r_{0,nn} \\
r_{1,11}   & r_{1,21}   & \ldots & r_{1,nn} \\
           &            & \ddots &          \\
r_{\ell+3,11} & r_{\ell+3,21} & \ldots & r_{\ell+3,nn}
\end{array}
\right]
\end{displaymath}

\begin{equation}
 =
\left[
\begin{array}{cccc}
\Re(q_{0,11})   & \Re(q_{0,21})   & \ldots & \Re(q_{0,nn}) \\
\Im(q_{0,11})   & \Im(q_{0,21})   & \ldots & \Im(q_{0,nn}) \\
\Re(q_{1,11})   & \Re(q_{1,21})   & \ldots & \Re(q_{1,nn}) \\
\Im(q_{1,11})   & \Im(q_{1,21})   & \ldots & \Im(q_{1,nn}) \\
                &                 & \ddots &          \\
\Re(q_{m,11}) & \Re(q_{m,21}) & \ldots & \Re(q_{m,nn}) \\
\Im(q_{m,11}) & \Im(q_{m,21}) & \ldots & \Im(q_{m,nn})
\end{array}
\right]
\end{equation}

Which is a set of real linear equations

\begin{equation}
\Matrix{AX} = \Matrix{B}
\end{equation}

where $\Matrix{A}$ is $(2m, 3+\ell)$, $\Matrix{X}$ is $(3+\ell, n^2)$,
and $\Matrix{B}$ is $(2m, n^2)$. These equations are overdetermined
if $2m > 3+\ell$, that is if the number of matrices $\Matrix{Q}_k$
is greater than the number of $\beta_i$ plus three divided by 2,
or the number of $\beta_i$ is less than two times the number of matrices
minus 3.

\begin{thebibliography}{}
\bibitem{roger} Roger, K.L., ``Airplane Math Modeling Methods for Active Control Design'', \textit{Proceedings of the 44th Meeting of the AGARD Structures and Materials Panel}, AGARD CP-228, April, 1977, pp. 4.1-4.11.
\end{thebibliography}
\end{document}
